\documentclass[10pt,letterpaper,draftclsnofoot,onecolumn]{IEEEtran}

\usepackage{graphicx}                                        
\usepackage{amssymb}                                         
\usepackage{amsmath}                                         
\usepackage{amsthm}                                          

\usepackage{alltt}                                           
\usepackage{float}
\usepackage{color}
\usepackage{url}

\usepackage{balance}
\usepackage[TABBOTCAP, tight]{subfigure}
\usepackage{enumitem}
\usepackage{pstricks, pst-node}

\usepackage{geometry}
\geometry{textheight=8.5in, textwidth=6in, margin=0.75in}
\usepackage[singlespacing]{setspace}

\newcommand{\cred}[1]{{\color{red}#1}}
\newcommand{\cblue}[1]{{\color{blue}#1}}

\usepackage{hyperref}
\usepackage{geometry}

\def\name{Group 11-05}

%pull in the necessary preamble matter for pygments output

%% The following metadata will show up in the PDF properties
\hypersetup{
  colorlinks = true,
  urlcolor = black,
  pdfauthor = {\name},
  pdfkeywords = {cs444},
  pdftitle = {CS 444 Assignment IV: The Slob Slab},
  pdfsubject = {CS 444 Assignment 4},
  pdfpagemode = UseNone
}

\begin{document}

\begin{titlepage}
\centering
{\Large Group 11-05: Daniel Schroeder, Brian Ozarowicz, and Luke Morrison\par}
\vspace{1cm}
{\scshape\Large CS 444 Operating Systems II\par}
{\scshape\Large Spring 2017\par}
\vspace{1cm}
{\huge\bfseries Assignment IV: The Slob Slab\par}
\vspace{2cm}
\begin{abstract}
This document is a summary of Assignment 4 for CS 444 Operating Systems II at Oregon State University Spring 2017. This document includes the design and implementation of the kernel assignment to implement the Slob Slab, responses to the design and implimenation questions for the kernel and concurrency assignments, and a work log.
\end{abstract}
\end{titlepage}

\section{Kernel Assignment}
\bigskip

\noindent\textbf{Design}
\medskip

\medskip

\noindent

\bigskip

\noindent\textbf{What do you think the main point of this assignment is?}
\medskip

\medskip

\noindent
\bigskip

\noindent\textbf{How did you personally approach the problem?}
\medskip

\medskip

\noindent
\medskip

\medskip

\noindent {Desired result}

\bigskip

\noindent\textbf{How did you ensure your solution was correct?}
\medskip

\medskip

\noindent

\bigskip

\noindent\textbf{What did you learn?}
\medskip

\medskip

\noindent everything.

\section{Concurrency Assignment}
\bigskip

\noindent\textbf{What do you think the main point of this assignment is?}
\medskip

\medskip

\noindent The main point of this concurrency assignment was to get more exposure to concurrent programming by implementing solutions to the Sleeping Barber Problem and a Mutual Exclusion problem. 

\bigskip

\subsection{The Sleeping Barber Problem}
\noindent\textbf{How did you personally approach the problem?}
\medskip

\medskip

\noindent Our solution to the Sleeping Barber Problem was an implementation of the solution found in The Little Book of Semaphores by Allen B. Downey with some slight variations.\\
We implemented a mutex that was used so customers could check how many customers were in the waiting room when they arrived. The problem required a MAX number of seats in the waiting room; so when this number was reached, an arriving customer left the store and did not receive a haircut.\\
From there, we implemented 4 semaphores:
\begin{itemize}
\item 1 for the barber's chair to ensure only one customer thread was being serviced at a time.
\item 1 semaphore was used to wake the barber and put him to sleep.
\item Then we had a semaphore that had a value equal to the number of chairs available in the waiting room that was incremented as a customer arrived and decremented once they left.
\item Our last semaphore was for the customer thread receiving a haircut to lock the thread until the barber thread was done ``cutting hair''.
\end{itemize}

\bigskip

\noindent\textbf{How did you ensure your solution was correct?}
\medskip

\noindent We used color coordinated printf statements for the consumer threads and barber thread to print out the logic of the problem. Each action performed by the threads was printed to the console in such a way that the user could follow all of the semaphore wait and posts and could understand what was happening throughout the program.\\
\medskip An example of our output is as follows:

{\fontfamily{pcr}\selectfont
\noindent \textcolor{green}{Barber done with haircut...}\\
\textcolor{red}{Customer 0 leaving barber shop.}\\
\textcolor{red}{Customer 8 is getting hair cut.}\\
\textcolor{green}{Barber is cutting hair for 4 seconds...}\\
\textcolor{green}{Barber done with haircut...}\\
\textcolor{red}{Customer 8 leaving barber shop.}\\
\textcolor{red}{Customer 2 is getting hair cut.}\\
\textcolor{green}{Barber is cutting hair for 7 seconds...}\\
\textcolor{green}{Barber done with haircut...}\\
\textcolor{red}{Customer 2 leaving barber shop.}\\
\textcolor{red}{Customer 3 is getting hair cut.}\\
\textcolor{green}{Barber is cutting hair for 3 seconds...}\\
\textcolor{green}{Barber done with haircut...}\\
\textcolor{red}{Customer 3 leaving barber shop.}\\
\textcolor{green}{All customers have been serviced!}\\
}
\bigskip

\noindent\textbf{What did you learn?}
\medskip

\medskip

\noindent We learned how to use semaphores with values other than 1 or 0 to lock critical sections after an arbitrary number of increments. In actuality, this program was of very similar fashion to the previous concurrency assignments and followed similar implementation schemes.
\medskip

\subsection{Mutual Exclusion Problem}
\noindent\textbf{How did you personally approach the problem?}
\medskip

\noindent For this problem, a mutually shared resources could be used by 3 threads at once, but no thread could begin using the resource until all three threads currently using were finished. To solve this problem, we used an is\_full flag to indicate that the resource was being used by 3 threads, and two int counters that counted the threads waiting for the resource and active threads using the resource. Our consumer thread function checks to see if there are three active threads using the resource and then sets a mutex to block the resource. As the threads finish their random consumption times, they decrement the active counter int and then release the mutex once all threads are finished.\\

\bigskip

\noindent\textbf{How did you ensure your solution was correct?}
\medskip

\noindent We used a series of printf statements that followed the program logic and printed everything that was happening in the program. For example, when a thread began using the resource, when a thread was blocked out and waiting, when the resource was full, etc... This allowed us to ensure that the program was functioning as the assignment required and allowed us to demo the program and show the TA how it was working.\\ 
\bigskip

\noindent\textbf{What did you learn?}
\medskip

\noindent We learned more about mutual exclusion of a shared resource in concurrent programming.
\medskip

\section{Version Control Log}
\bigskip

\noindent\begin{tabular}{l l l}\textbf{Detail} & \textbf{Author} & \textbf{Description}\\\hline
\href{https://github.com/DSchroederOSU/CS444-Group\_11\_05/commit/31d0e499089bc16ec0f1a8651c6579b882fe3837}{31d0e49} & ozarowib & HW2 preparation\\\hline
\href{https://github.com/DSchroederOSU/CS444-Group\_11\_05/commit/d76f94b4ef4e153d4a4082e754008b003cbcb8fc}{d76f94b} & Luke Morrison & luke's philosophers\\\hline
\href{https://github.com/DSchroederOSU/CS444-Group\_11\_05/commit/6ecb0b1be6e85c16b4d50d2c492c10fb3598a1d2}{6ecb0b1} & Luke Morrison & luke philosophers\\\hline
\href{https://github.com/DSchroederOSU/CS444-Group\_11\_05/commit/8fe1537eedfae5e846f6cb43d280fc8f716fc7d0}{8fe1537} & Luke Morrison & howto for changing scheduler\\\hline
\href{https://github.com/DSchroederOSU/CS444-Group\_11\_05/commit/1858a4b72e56d55cf51047d145d37da8924d0a95}{1858a4b} & DSchroederOSU & fixed concurrency\\\hline
\href{https://github.com/DSchroederOSU/CS444-Group\_11\_05/commit/fe2fee9cb7be6a8dc4f24a05b72d855ab6f14d2e}{fe2fee9} & DSchroederOSU & added tex and make\\\hline
\href{https://github.com/DSchroederOSU/CS444-Group\_11\_05/commit/f1cc0bee6c166599c52da86c7462076456200697}{f1cc0be} & Luke Morrison & cleanup + fix issues\\\hline
\href{https://github.com/DSchroederOSU/CS444-Group\_11\_05/commit/3b97f026eb869e0905fb8d1b09213294b9f629d6}{3b97f02} & Luke Morrison & test for IO scheduler\\\hline
\href{https://github.com/DSchroederOSU/CS444-Group\_11\_05/commit/a4abfb8d116f312257ec679de7d846a52eec9281}{a4abfb8} & Luke Morrison & added command line arguments\\\hline
\href{https://github.com/DSchroederOSU/CS444-Group\_11\_05/commit/b1e7d32a4b99257f3570ea84c4e2a08c60eb89e1}{b1e7d32} & DSchroederOSU & fixed concurrency\\\hline
\href{https://github.com/DSchroederOSU/CS444-Group\_11\_05/commit/df9f80af0d69e84ff3f2b052b8af192c2340c0d5}{df9f80a} & DSchroederOSU & tex file and concurrency work\\\hline
\href{https://github.com/DSchroederOSU/CS444-Group\_11\_05/commit/93ea155de2ddeeb5360af8ec1df1ad7b1b315a10}{93ea155} & DSchroederOSU & adding tex file\\\hline
\href{https://github.com/DSchroederOSU/CS444-Group\_11\_05/commit/2fdcd23f98fb16f5f5cc05d36e1e46708aaf2463}{2fdcd23} & DSchroederOSU & working on code in Latex file\\\hline
\href{https://github.com/DSchroederOSU/CS444-Group\_11\_05/commit/8df401398a214f996bbdd6e5ef42407e2f5d33b5}{8df4013} & ozarowib & conformed to class style\\\hline
\href{https://github.com/DSchroederOSU/CS444-Group\_11\_05/commit/38ad76e091720598addf73749757dffbc01ea642}{38ad76e} & DSchroederOSU & added code to tex file\\\hline
\href{https://github.com/DSchroederOSU/CS444-Group\_11\_05/commit/dc6e9301c617451dcba6073b38c9c5a406482b20}{dc6e930} & DSchroederOSU & Merge branch 'master' of https://github.com/DSchroederOSU/CS444-Group\_11\_05\\\hline\end{tabular}
\bigskip

\section{Work Log}
\bigskip

\begin{itemize}
\item Brian copied noop-iosched.c to sstf-iosched.c in the /block directory
\item Brian researched schedulers and elevator algorithms, started kernel assignment
\item Group completed the concurrency program
\item Luke wrote a python script for generating I/O to test new scheduling algorithm
\item Daniel added .tex and Makefile to the project repository
\item Group continued work and research on the kernel assignment
\item Luke and Brian implemented final fixes in sstf-iosched.c
\item Group finished the assignment write-up in a Google Doc
\item Brian transferred the content from the Google Doc write-up to the \LaTeX\ file
\item Group pushed final changes to GitHub and prepared tarball for submission on TEACH
\end{itemize}

\end{document}
